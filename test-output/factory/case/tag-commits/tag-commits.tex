\subsubsection{Test Case: Tag Commits}
\begin{description}[align=right,leftmargin=3.2cm,labelindent=3.0cm]
\item[Purpose:] This teset case is used to demonstrate Git's abbility to tag a commit.  A Git repository has been pre-seeded with the following tags. \textbackslashbegin\{longtable\} \{\textbarC\{1.2cm\}\textbarL\{10cm\}\textbar\} \textbackslashhline Tag \& Git Checksum\textbackslash\textbackslash \textbackslashhline v1.0 \& bcde6bf1ed2841f39e719ccd323df4cc33497e04 \textbackslash\textbackslash \textbackslashhline v1.1 \& b64d30b59ff026f12350a5c040d779a5a9a59214 \textbackslash\textbackslash \textbackslashhline v1.2 \& 8aa0308a4b661e97df4002ccd7f0ee3dd2f55994 \textbackslash\textbackslash \textbackslashhline v2.0 \& ea06210d6f84c2daf26f00c39643a9cd9b98a4a5 \textbackslash\textbackslash \textbackslashhline v2.1 \& 440ba8b56feb5941783b9cb57ed192391cd251c6 \textbackslash\textbackslash \textbackslashhline v2.2 \& c47279b87ed77413792908818215f53364567bb1 \textbackslash\textbackslash \textbackslashhline \textbackslashend\{longtable\}  A Git Checksum is unique to a Git installation and are listed here as examples.  In the event these tags do not exist, the following commands are used to recreate them and publish them to the git server.      \textbackslashtextbf\{git tag v1.0 [git-checksum]\}  where git-checksum is the git-commit-id you are associating the tags with.  Git-commit-id's are globally-uniquie.      \textbackslashtextbf\{git push origin --tags\}
\item[Requirement:] IUR-002, IUR-003, IUR-004, IUR-006, IUR-007, IUR-008, and IUR-009.
\end{description}
